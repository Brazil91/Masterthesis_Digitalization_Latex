\newpage
\begin{table}[hptb!]
\caption{Benötigte LabVIEW Konfigurationsfunktionen Teil 1}
\begin{center}

\begin{tabularx}{1\textwidth}{m{0.5cm}m{4.5cm}X}
\hline
Nr. & Funktion & Lokalisation der Funktion in LabVIEW \\
\hline
1 & For Loop & BD/Functions/Programming/Structures \\
2 & Case Structures & BD/Functions/Programming/Structures \\
3 & First Call? & BD/Functions/Programming/Synchronization \\
4 & Configure Port & BD/Functions/Instruments I/O/Serial \\
5 & Bytes at port & BD/Functions/Instruments I/O/Serial \\
6 & Read & BD/Functions/Instruments I/O/Serial \\
7 & Write & BD/Functions/Instruments I/O/Serial \\
8 & Close & BD/Functions/Instruments I/O/Serial \\
9 & Flush Buffer & BD/Functions/Instruments I/O/Serial \\
10 & String Konstante & BD/rM auf Funktionseingang/Create Constant \\
11 & Numerische Constante & BD/rM auf Funktionseingang/Create Constant \\
12 & String Indicator & FP/Modern/String \& Path \\
13 & VISA ressource & FP/Modern/I/O \\
14 & Stop Button & FP/Boolean \\
15 & VI Object I/O Definition & obere rechte Ecke des LabVIEW Fensters \\
16 & Serial Sub VI & drag and drop (siehe Abbilding \ref{step4} \\
17 & Concatenate Strings & BD/Functions/Programming/Strings \\
18 & Iterationszzähler & siehe Abbildung \ref{step6} \\
19 & Shift Register & siehe Abbildung \ref{step6}  \\
20 & While Loop \newline \mbox{Abbruchbedingung} &  siehe Abbildung \ref{step6} \\
21 & String Indicator/Anzeige &\makecell[l]{ FP/Modern/String \& Path/String Indicator oder\\ rM auf Datenquelle/Create Indicator}\\
22 & String Control/Eingabe & \makecell[l]{ FP/Modern/String \& Path/String Indicator oder\\ rM auf Datenquelle/Create Control}\\
23 & Search/Split String & BD/Functions/Programming/String/Additional String Functions\\
24 & Case Selektor & siehe Abbildung \ref{step7} \\
25 & String Subset & BD/Functions/Programming/String \\
26 & Equal? & BD/Function/Programming/Comparison \\
27 & Not & BD/Function/Programming/boolean \\
28 & Regular Expression Match & BD/Functions/Programming/String \\
29 & Local variable & BD/Functions/Structures \\




\hline
\end{tabularx}
\end{center}
\label{tab:labviewserialobject}
\end{table}

\pagebreak

