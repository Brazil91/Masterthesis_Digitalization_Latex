\chead{\headmark}
\automark[subsection]{section}

\addsec{Abkürzungsverzeichnis}
%\thispagestyle{empty}
%\addcontentsline{toc}{section}{Abkürzungsverzeichnis}

%Seitenzähler reset mit große römische Ziffern
%%_____________________________________________________________________
\pagenumbering{Roman}
%%_____________________________________________________________________


%\usepackage{acronym}

%\begin{acronym} 
%
%%\ac{abktag}   abkürzung wird eingeführt
%%\acs{abktag} Abkürzung wird nicht eingeführt
%%\acl{abktag} gibt die ausgeschriebene Form aus
%
%\acro{KI}[KI]{künstliche Intelligenz} 
%\acro{mvt}[mvt]{mechanische Verfahrenstechnik}
%
%\end{acronym}

% Wörter untrennbar machen:
% \mbox{Untrennbar}

\begin{large}
\begin{tabbing}
\hspace{90pt}\=\hspace{200pt}\=\kill
AI \> artificial intelligence \\
ANSI \> American National Standards Institute \\
ASCII \> American Standard Code for Information Interchange \\
Bd \> Baudrate \\
BD \> Blockdiagramm \\
BD \> Blockdiagramm \\
BuB \> Bedinung und Beobachtung \\ 
Com \> Serieller Anschluss unter Windows \\
CTS \> Clear to Send \\
DAQ \> data acquisition, Datenakquisition \\
DBMS \> Datenbank Management Systemen \\
DEE \> Datenendeinrichtung \\
DÜE \> Datenübertragungseinrichtung \\
EIA \> Electronic Industrial Alliance \\
ELN \> Electronic Laboratory Notebook \\
ERP \> Enterprise Resource Planning \\
FP \> Front Panel \\
FP \> Front Panel \\
GND \> Signal Ground \\
HMI \> Human-Machine-Interface \\
ICS \> Industrial Control System (deutsch Industrielle Kontrollsysteme) \\
IMS \> Informationsmanagementsysteme \\
IoT \> Internet of Things \\
ISA \> International Society of Automation \\
ISO \> International Organization for Standardization \\
KI \> künstliche Intelligenz \\
LIMS \> Labor Management Informations Systeme  \\
MES \> Manufacturing Excecution System\\
MESA \> Manufacturing Enterprise Solutions Association \\
MOM \> Manufacturing Operations Management\\
MVT \> mechanische Verfahrenstechnik \\
PERA \> Purdue Enterprise Reference Architecture \\
rM \> rechte Maustaste \\
RS-232 \> Recommended Standard oder Radio Section 232 \\
RTS \> Request to Send \\
RxD \> Receive Data \\
SETSVI \> Simple Elapsed Time Sub VI \\ 
SKDM \> Schwebekörperdurchflussmesser \\
SQL \> Structured Query Language \\
SSVI \> Serial Sub VI \\
TxD \> Transmit Data \\
USB \> Universal Serial Bus \\
VI \> Virtuelles Instrument \\
VT \> Verfahrenstechnik \\
WBF \> World Batch Forum\\
WSN \> Wireless Sensor Network \\
XML \> Extensible Markup Language \\
XSD \> [XML] Schema Definition \\




\end{tabbing} 
\end{large}


\pagebreak

\addsec{Symbolverzeichnis}
%\thispagestyle{empty}


\vspace{0.5em} 
\begin{large}
\begin{tabbing}
\hspace{90pt}\=\hspace{300pt}\=\kill

\textbf{Symbol} \> \textbf{Bezeichnung} \> \textbf{Einheit} \\[0.3em] 
$A$ \> Anströmfläche \> $\mathrm{m^2}$ \\
$ \vec{a}$ \> Beschleunigung \> $\mathrm{m/s^2}$\\
Bd \> Baudrate \> Byte/s  \\
$c_p$ \> Wärmekapazität \> $\mathrm{kJ / kg \cdot K}$\\
$c_w$ \> Widerstandsbeiwert \> - \\
$d$ \> Durchmesser \> $\mathrm{mm^{2}}$ \\
$\varepsilon$ \> Porösität \> - \\
$\eta$ \> dynamische Viskosität \> $\mathrm{Pa s}$ \\
$\mathit{Eu}$ \> Eulergleichung \\
$F$ \> Kraft \> $\mathrm{N}$ \\
$ \vec{F}$ \> Kraftvector \> $\mathrm{kg \cdot m/s^2}$ \\
$F$ \> Kraft \> $\mathrm{N}$ \\
$g$ \> Gravitationskonstante \>  $\mathrm{m/s^2}$\\
$I$ \> Stromstärke \>$ \mathrm{A}$ \\
$k$ \> Dehnungsempfindlichkeit \> - \\
$m$ \> Masse \> $\mathrm{kg}$ \\
$\nu$  \> Querkontraktionszahl (Poisson-Konstante) \> - \\
$P, p$ \> Druck \> $\mathrm{N/m^2}$ \\
$\psi$ \> Dehnung \> - \\
$\pi_{l,t}$ \> longitudinale/transversale Piezokonstante \> $\mathrm{Pa^{−1}}$ \\
$R$ \> elektrischer Widerstand \> $\Omega$ \\
Re \> Reynoldszahl \> - \\
$\rho$ \> Dichte \> $\mathrm{kg / m^3}$ \\
$\rho_{el}$ \> spezifischer elektrischer Widerstand \> $\Omega~ \mathrm{ \cdot~ m}$ \\
$S$ \> Surface \> $\mathrm{mm^2}$ \\
$s$ \> Filterkuchendicke \\
$\sigma_{l,t}$ \> longi./transv. mechanische Spannungskomponente \> $\mathrm{Pa}$ \\
$T$ \> Temperatur \> $\mathrm{K}$ \\
$U$ \> elektrische Spannung \> $\mathrm{V}$ \\
$\dot{V}$ \> Volumenstrom \> $\mathrm{m^2/s}$ \\
$v$ \> Geschwindigkeit, Leerrohrgeschwindigkeit \> $\mathrm{m/s}$ \\
$w$ \> charakteristische Geschwindigkeit \> $\mathrm{m/s}$ \\
$x$ \> Feinheitsmerkmal \> $\mathrm{mm}$\\


\end{tabbing} 
\end{large}

\newpage
\addsec{Indexverzeichnis}
%\thispagestyle{empty}


\vspace{0.5em} 
\begin{large}
\begin{tabbing}
\hspace{90pt}\=\hspace{200pt}\=\kill

\textbf{Index} \> \textbf{Bezeichnung} \\[0.3em] 
$\mathrm{32}$ \> Sauter(-durchmesser) \\
$\mathrm{A}$ \> Auftrieb \\
$\mathrm{a}$ \> Anström(-fläche) \\
$\mathrm{Br}$ \> Wheatstone Brücke \\
$\mathrm{c}$ \> Coriolis \\
$\mathrm{el}$ \> elektrisch \\
$\mathrm{f}$ \> Fluid \\
$\mathrm{fb}$ \> Festbett \\
$\mathrm{FM}$ \> Filtermittel \\
$\mathrm{fs}$ \> Feststoff \\
$\mathrm{G}$ \> Gravitation \\
$\mathrm{h}$ \> hydraulisch \\
$\mathrm{irr}$ \> irreversibel \\
$\mathrm{K}$ \> Filterkuchen \\
$\mathrm{konst}$ \> konstant \\
$\mathrm{L}$ \> Länge \\
$\mathrm{l}$ \> longitudinal \\
$\mathrm{m}$ \> Massen(-konzentration) \\
$\mathrm{mf}$ \> minimal Fluidisation \\
$\mathrm{Q}$ \> Querkontraktion \\
$\mathrm{s}$ \> Sink(-geschwindigkeit) \\
$\mathrm{t}$ \> transversal  \\
$\mathrm{V}$ \> Volumen \\
${\mathrm{vs}}$ \> Volumen(-konzentration) \\
$\mathrm{W, w}$ \> Widerstand \\
$\mathrm{ws}$ \> Wirbelschicht \\
$\mathrm{x}$ \> Feinheitsmerkmal \\


\end{tabbing} 
\end{large}
\pagebreak


\addsec{Glossar englischer Begriffe}


\vspace{0.5em} 
\begin{large}
\begin{tabbing}
\hspace{220pt} \=\hspace{200pt}\=\kill

\textbf{Englisch} \> \textbf{Deutsch} \\[0.3em] 
\raggedright AI \> künstliche Intelligenz \\
classification \> Klassifizieren \\
DAQ \> Datenakquisition \\
digital Twin \> digitaler Zwilling \\
dryfreezer \> Gefriertrockner \\
dustfiltration \> Staubabscheidung \\
fixed-bed \> Festbett \\
fluidization  \> Fludisierung \\
high pressure homogenizer \> Hochdruckhomogenisator \\
laser diffraction device \> Laserbeugerpartikelanalysator \\
machien learning \> maschinelles Lernen \\
particle bed, fixed-bed filtration \> Festbett-/ Kuchenfiltration \\
unit operation \> verfahrenstechnische Grundoperation \\
principal \> Auftraggeber $\Rightarrow$ Dozenten \\
rollcrusher \> Walzenbrecher \\
sieveing \> Sieben \\
standard operating procedures (SOP) \> Handlungsanweisung, Versuchsanweisung \\
supervisor \> Schichtleiter $\Rightarrow$ wissenschaftlicher Mitarbeiter \\
tensil tester \> Zugversuchapparatur \\
\end{tabbing} 
\end{large}

\pagebreak