% Tikz, Chains Beispiel

%\begin{figure}
%\begin{center}
%\caption{PERA Model, angelehnt an \, \cite{pera_modell, ics_kompendium}}
%\vspace{1em}
%\begin{tikzpicture}[
%	scale=0.75,
%	start chain=1 going below, 
%	start chain=2 going right,
%	node distance=1mm,
%	desc/.style={
%		scale=0.75,
%		on chain=2,
%		rectangle,
%		rounded corners,
%		draw=black, 
%		very thick,
%		text centered,
%		text width=8cm,
%		minimum height=12mm,
%		fill=blue!30
%		},
%	it/.style={
%		fill=blue!10
%	},
%	level/.style={
%		scale=0.75,
%		on chain=1,
%		minimum height=12mm,
%		text width=2cm,
%		text centered
%	},
%	every node/.style={font=\sffamily}
%]
%
%% Levels
%\node [level] (Level 5) {Level 5};
%\node [level] (Level 4) {Level 4};
%\node [level] (Level 3) {Level 3};
%\node [level] (Level 2) {Level 2};
%\node [level] (Level 1.5) { };
%\node [level] (Level 1) {Level 1};
%\node [level] (Level 0) {Level 0};
%
%% Descriptions
%\chainin (Level 5); % Start right of Level 5
%% IT levels
%\node [desc, it] (Archives) {Archives/File Servers};
%\node [desc, it, continue chain=going below] (ERP) {Produktionsführung wie ERP/Finance/Messaging};
%% ICS levels
%\node [desc] (Operations) {Operatives Management};
%\node [desc] (Supervisory) {Real Time Prozessführung};
%\node [desc, text width=3.5cm, xshift=2.25cm] (PLC) {PLC/RTU IP Communication};
%%\node [desc, text width=3.5cm, xshift=-4.5cm] (SIS) {Safety Instrumented Systems};
%\node [desc, xshift=-2.25cm] (SIS) {Prozessführung im Feld};
%\node [desc] (IO) {I/O from Sensors};
%
%\end{tikzpicture}
%\label{pera_modell}
%\end{center}
%\end{figure}