\markboth{Abstract}{Zusammenfassung/Abstract}


\begin{abstract}
Im Verlauf des Projekts wurden Konzepte zur Digitalisierung des verfahrenstechnischen Labors erstellt. Es sind zwei Konzepte erstellt worden, die aufeinander aufbauen. Das erste Konzept mit dem Titel \glqq Konzept 3.0\grqq{} beschreibt die Generierung von Messwertdaten mittels Sensoren und die softwaretechnische Verarbeitung und Archivierung. Das zweite Konzept, mit dem Titel \glqq Konzept 4.0\grqq{} beschreibt die zukünftige, potentielle Implementierung weitere Applikationen im Rahmen von Big Data. Da das zweite Konzept Datenbankmanagementsysteme im Zentrum hat, ist ein Datenbankkonzept, angelehnt an den ISA-S95 and B2MML, erstellt und dieses anschließend auf ein relationales Datenbankkonzept transferiert worden. Abschließend wurden zwei Versuchsstände, die Wirbelschichtversuchsanlage und die Filterkuchenversuchsanlage, mit digitaler Sensorik ausgestattet und Programme in LabVIEW geschrieben, die die Datenströme beider Versuchsstände verarbeiten können und Tabstoppgetrennte \,{\Menlo *.txt}-Dateien generieren.
\end{abstract}


\vspace{2em}


\selectlanguage{english}


\begin{abstract}
In the course of the project, concepts for digitizing the process engineering laboratory were created. Two concepts have been created. The second one build on top of the first one. The first concept, entitled Concept 3.0, describes the generation of measured value data by means of sensors and the software processing and storage. The second concept, titled Concept 4.0, describes the future, potential implementation of further applications in the context of Big Data. Since the second concept has database management systems at its core, a database concept, based on ISA-S95 and B2MML, has been created and then transferred to a relational database concept. Finally two test rigs, the fluidization bed test rig and the filtration (fixed-bed) test rig, have been equipped with digital sensors. Programs have been written in LabVIEW that can process the data streams of both test rigs and generate tab-stop separated \,{\Menlo *.txt} files.
\end{abstract}

\selectlanguage{ngerman}
\pagebreak

\automark[subsection]{section}