%\begin{table}[b!p]
%\caption{Benötigte LabVIEW Konfigurationsfunktionen Teil 2}
%\begin{center}
%
%\begin{tabularx}{1\textwidth}{m{0.5cm}m{4.5cm}X}
%\hline
%
%35 & wait until next ms multiple & BD/Functions/Programming/Timing \\
%36 & Tick Count (ms) & BD/Functions/Programming/Timing \\
%37 & Boolean Control & rM/Create Control \\
%38 & Write to Text & BD/Functions/Programming/File I/O \\
%39 & Set File Position & BD/Functions/Programming/File I/O/Adv File Func \\
%40 & Build Path & BD/Functions/Programming/File I/O \\
%41 & Close File & BD/Functions/Programming/File I/O \\
%42 & Open/Create/Replace File & BD/Functions/Programming/File I/O \\
%43 & Elapsed Time & BD/Functions/Programming/Timing \\
%44 & VTP Elapsed Time & \ref{fig:simple_elapsed_time}
%45 & And, Or & BD/Function/Programming/boolean \\
%
%\end{tabularx}
%\end{center}
%\label{tab:labviewserialobject2}
%\end{table}


\begin{table}[b!p] % Benötigte_LabVIEW_Konfigurationsfunktionen_Teil_2
\caption{Benötigte LabVIEW Konfigurationsfunktionen Teil 2}
\begin{center}
\begin{tabularx}{1\textwidth}{m{0.5cm}m{4.5cm}X}
\hline
Nr. & Funktion & Lokalisation der Funktion in LabVIEW \\
\hline
30 & LED & \makecell[l]{BD/rM auf Objektausgang oder\\ Datenleitung/Create Indicator} \\
31 & Format Date/Time String & BD/Functions/Programming/Timing \\
32 & Tabulator & BD/Functions/Programming/String \\
33 & Linefeed/Newline & BD/Functions/Programming/String \\
34 & wait (ms) & BD/Functions/Programming/Timing \\
35 & wait until next ms multiple & BD/Functions/Programming/Timing \\
36 & Tick Count (ms) & BD/Functions/Programming/Timing \\
37 & Boolean Control & rM/Create Control \\
38 & Write to Text & BD/Functions/Programming/File I/O \\
39 & Set File Position & \makecell[l]{BD/Functions/Programming/\\File I/O/Adv File Func} \\
40 & Build Path & BD/Functions/Programming/File I/O \\
41 & Close File & BD/Functions/Programming/File I/O \\
42 & Open/Create/Replace File & BD/Functions/Programming/File I/O \\
43 & Elapsed Time & BD/Functions/Programming/Timing \\
44 & VTP Elapsed Time & Abbildung \ref{fig:vtp_elapsed_time} \\ 
45 & String Lenght & BD/Functions/Programming/String \\
46 & Or & BD/Function/Programming/Boolean \\ 
47 & Greater? & BD/Function/Programming/Comparison \\
%48 & \makecell[l]{\textbullet \,  potentieller Informationsverlust von \\ \hspace{0,19cm} vermeintlich unwichtigen Daten}  & bla \\

\end{tabularx}
\end{center}
\label{tab:labviewserialobject2}
\end{table} 