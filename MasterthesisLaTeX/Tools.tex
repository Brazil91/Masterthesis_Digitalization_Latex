\addsec{Tools}

 Alle Tools die im Verlauf des Projekts eingesetzt wurden sind in der folgenden Liste aufgelistet:


\begin{itemize}
\item Hardware
	\begin{itemize}
	\item Smartphone P30+ (privat) für die Fotodokumentation
	\item MacBook Pro 2011 (privat)
	\item Windows 10 Desktop PC (privat)
	\item Windows 10 Desktop PC
	\end{itemize}

\item Software
	\begin{itemize}
	\item Powerpoint für Präsentation, Grafikerstellung (z.B. elektrisches Schaltschema, Datenbank) und Bildmanipulationen.
	\item PDF24 
		\begin{itemize}	
		\item zum beschneiden der PDF Ränder
		\item Zum "Drucken" von Dateien in ein PDF, anstelle eines Papiers
		\end{itemize}
	\item Tabellenkalkulationsprogramm: Excel
	\item Bildmanipulation: GIMP 
	\item Textsatz: \LaTeX zur Anfertigung des Dokuments
		\begin{itemize}
		\item[-] Fonts (XeLaTeX notwendig, um die Fonts nutzen zu können)
			\begin{itemize}			
			\item[1] Arno Pro
			\item[{\Hypatia 2}] {\Hypatia Hypatia Sans}
			\item[{\Menlo 3}] {\Menlo Menlo (Python font)}
			\end{itemize}
		\item[-] Bibliotheksmanagement: JabRef 
		\item Für Formeln im Hauptdokument wurden die folgenden Normen beachtet:
		\begin{itemize}
		\item DIN 1338 Formelschreibweise und Formelsatz
		\item DIN 1304 Formelzeichen
		\item DIN 1301 Einheitennamen, Einheitenzeichen
		\item DIN 1313 Größen 
		\end{itemize}		
		\end{itemize}
	\item LabVIEW 2019 auf den HMI`s an der HAW Life Sciences	
	\item LabVIEW 2020 auf dem MacBook pro und dem privatem PC
	\item LabVIEW 2021 auf den HMI`s an der HAW Life Sciences
	\item Anaconda als Python Distribution
		\begin{itemize}
		\item Spyder als Python Editor
		\end{itemize}
\end{itemize}
\end{itemize}

