\subsection{Messtechnik}

In diesem Abschnitt werden ausgewählte analoge und digitale Druck- und Volumenstrommesstechniken erläutert, die in Laboren anwendung finden.

\subsubsection{Gas Durchflussmessung}

Den Durchfluss eines Mediums in einer geschlossenen Rohrleitung lässt sich mit verschiedenen Messprinzipien bestimmen. Zu den Volumendurchflussmessungen gehören z.B. die:

\begin{itemize}
\item Ultraschalldurchflussmessung
\item Wirkdruckverfahren
\item Schwebekörperdurchflussmesser
\end{itemize}

und zu den Massendurchflussmessprinzipien zählen z.B. die:

\begin{itemize}
\item thermische Massendurchflussmessung
\item Coriolis Durchflussmessung
\item Mikromechanische Flusssensormessung
\end{itemize}

%\paragraph{Volumendurchflussmessung}

%\subparagraph{Wirkdruckverfahren}

%\subparagraph{Ultraschalldurchflussmessung}

\paragraph{Massendurchflussmessung}

Mittels der 

\subparagraph{Coriolis}

Massendurchflussmesser mini CORI-FLOW™ series

\subparagraph{thermische Massendurchflussmessung}

Die Durchflussmessung von Gasen ist mittels thermischen Massendurchflussmesser möglich. Zur Messung des Massendurchflusses wird die Wärmeleitfähigkeit von Fluiden genutzt. Der prinzipielle Aufbau so eines Sensors ist in Abbildung \ref{MFC} dargestellt.

\bild{0.5}
{MFC.png}
{0em}
{Prinzip eines thermischen Massendurchflussensors}
{Prinzip eines thermischen Massendurchflussensors}
{MFC}

 Im inneren des Durchflussmessers sind zwei Temperatursensoren verbaut. Ein Sensor befindet sich nicht direkt im Strömungskanal (in einem Sackloch) und misst die Veränderung der Umgebungstemperatur $T_U$, als auch die Wärmeübertragung des Fluids als Referenz. Der zweite Sensor wird beheizt. Die Temperaturdifferenz $\Delta T$ wird konstant gehalten. Je höher die Fließgeschwindigkeit des Mediums, desto mehr Energie muss dem zweiten Sensor zugeführt werden, um die Temperaturdifferenz konstant zu halten.  Der Massenstrom $\dot{m}$ ist dem Wärmestrom $\dot{Q}$ proportional. Es gilt folgender Zusammenhang zwischen Massen- und Wärmestrom:

\begin{align}
\dot{Q}=\dot{m} \cdot c_p \cdot  \Delta T
\end{align}

Der Massenstrom ist der zugeführten Energie in Form von Strom $I^2$ proportional. Der zweite Sensor hat einen Temperaturabhängigen Widerstand $R(T)$. 

\begin{align}
\dot{m}=\dfrac{R(T) \cdot I^2}{c_p \cdot \Delta T}
\end{align}

Beide Sensoren sind zu einer Wheatstone'schen Messbrücke verschaltet, $\sqrt{\dot{m}}$ ist der Ausgangsspannung $\Delta U_Br$ proportional .

\subsubsection{Druckmessung}

